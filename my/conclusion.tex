% !TeX root = ../main.tex
\cleardoublepage
\chapter{结论}\label{ch:conclusion}


\section{论文的主要成果与创新点 }\label{sec:conclusion-review}

本文通过分析现有静电卡盘静电力检测方案中存在的检测准确性与可重复性等问题,自主提出了一种改进方案,设计并搭建了一套完整的静电卡盘静电力检测软硬件平台,提出了一种利用该平台自动化检测静电力的方法及其规范流程。通过分析气体背吹压力与微力探头受力的相关性,深入探讨了晶圆多种脱附过程的作用机理以及脱附时刻的判断方法。本研究的主要成果与创新点如下:

\begin{enumerate}
  \item 针对背吹平衡法中气隙扩大、晶圆变形等影响静电力检测准确性的诸多不利因素,提出一种利用微力探头检测、控制并消除气隙变化的方法,可准确地判断脱附时刻,并在一定程度上揭示了晶圆脱附过程的复杂性。同时提出了一种利用重力平衡原理标定背吹压强等效作用面积的方法,为进一步探究静电卡盘中复合场耦合作用提供支持。
  \item
  已有研究对晶圆脱附过程作出诸多假设,如静电力随间隙增加而降低、背吹气层在晶圆下表面压强均匀分布等,而本文实验研究表明晶圆实际脱附过程远比假设复杂的多,本文对实际测得的脱附过程进行了归纳与解释,并深入探讨了判断脱附时刻的一般性方法。
  \item 为提高检测准确性和检测效率,本文设计并实现了静电卡盘静电力检测流程的电控与自动化,可在较短时间内获得大量有效数据,为深入研究静电力产生与消除机理提供了必要的软硬件基础。
\end{enumerate}



\section{工作展望}\label{sec:conclusion-future}

由于时间、条件所限,本研究中提出的检测方案仍有改进余地。首先,如\ref{sec:exp-main-proc}节所述,试验条件若能全部实现,预计试验的可重复性与准确度均会有较大提升。检测方案中的重力平衡法未能实际实现,其主要原因是因为发现“压强作用等效面积”在复杂的脱附过程中可能并非常量,而可能与具体的脱附过程有关,导致标定该量的方法需要做进一步研究。

检测平台的设计与实现也有较大的改进空间:检测平台中的背吹控制系统仍不完善,现在完全依靠机械式减压阀来供压,应在未来的工作中,选用更合适的电子比例阀或压力控制器,甚至实现反馈控制泵式压力控制器等,以实现背吹供压的自动化。另外,由于本研究带有探索性质,电控系统未能实现更高的集成度,目前仍是较多模块松散连接的方式,可以考虑在检测平台定性之后,按照本文中的原理方案,改进电控系统的具体设计与实现。

由于脱附过程极为复杂,仅靠一个微力探头得出的数据难以反映整个晶圆面积上发生的物理过程全貌,考虑在后续试验中,在圆周对称的位置上加装微力探头,或增加激光测距仪/干涉仪等,获得更多关于晶圆在脱附过程中局部受力与形变的数据,以此加深对脱附过程的理解。

现有的商业有限元仿真软件(ANSYS、COMSOL等)均无法较好地建立同时考虑机械接触与背吹气体流动的静电卡盘力学模型,因此本工作未能包括试验数据与仿真结果的对比研究。若能开发出准确建模该系统的仿真方法,则可更好地揭示脱附过程中整个系统内部发生的不易观察、检测的物理过程,获得对晶圆脱附、吸附、乃至整个静电卡盘设计优化的更清楚的认识。