% !TeX root = ../main.tex
\cleardoublepage
\chapter{结论}\label{ch:conclusion}


\section{论文的主要工作和创新}\label{sec:conclusion-review}

本文通过对已有静电卡盘静电力检测方案的分析,找出了其中存在的影响其检测准确度与可重复性的问题,并针对这些问题提出了改进思路,形成了改进的静电力检测方案。依据改进方案,本文设计了完整的大气条件下的静电力检测平台(可推广至真空条件),并将其实际搭建出来,经过初步试验,确定了使用该平台检测静电力的方法、流程与注意事项,将其应用于正式试验中,得到了大量能够反映静电卡盘静电吸附特性的数据,成功验证了检测方案以及平台设计能够较为准确地检测静电力。本研究主要的创新点如下:

\begin{enumerate}
  \item 提出在已有的背吹平衡法检测静电力的方案中,气隙在晶圆脱附过程中的扩大以及背吹气体的稳态流动与等效作用面积对检测的准确性产生的影响。针对气隙影响,提出了微力探头这一新检测思路,既能控制、消除气隙扩大的影响,又能更加准确地判断脱附,还部分揭示出了晶圆脱附过程的复杂性。针对背吹气体,提出了重力平衡法这一思路,可作为进一步探究晶圆受背吹作用提供借鉴。
  \item
  晶圆脱附过程的复杂性实际上推翻了已有研究中对其作出的多种假设(如静电力一定随间隙增加而降低、背吹气层在晶圆下表面的压强为均匀分布等),为相关研究指出了可能的新方向。
  \item 已有的静电力检测方案受其环境影响,费时费力;本文提出的检测方案直接在大气下工作,且检测流程中大部分操作已实现电控与自动化,可在较短时间内获得大量有效数据,不仅提高了检测的精确度,还为今后对静电力产生与消除机理的研究提供了数据基础。
\end{enumerate}


\section{工作展望}\label{sec:conclusion-future}

由于时间、条件所限,本研究中提出的检测方案未能全部付诸实践。首先,如\ref{sec:exp-main-proc}节所述,试验条件仍有较大的改进的余地,若能全部实现,预计试验的可重复性与准确度均会有较大提升。检测方案中的重力平衡法未能实际实现,其主要原因是因为发现“压强作用等效面积”在复杂的脱附过程中可能并非常量,而可能是与具体的脱附过程有关,导致标定该量失去意义。

检测平台的设计与实现也有较大的改进空间:检测平台中的背吹控制系统仍不完善,现在完全依靠机械式减压阀来供压,应在未来的工作中,选用更合适的电子比例阀或压力控制器,甚至实现反馈控制泵式压力控制器等,以实现背吹供压的自动化。另外,由于本研究带有探索性质,电控系统未能实现更高的集成度,目前仍是较多模块松散连接的方式,可以考虑在检测平台定性之后,按照本文中的原理方案,改进电控系统的具体设计与实现。

由于脱附过程极为复杂,仅靠一个微力探头得出的数据难以反映整个晶圆面积上发生的物理过程全貌,考虑在后续试验中,在圆周对称的位置上加装微力探头,或增加激光测距仪/干涉仪等,获得更多关于晶圆在脱附过程中局部受力与形变的数据,以此加深对脱附过程的理解。

现有的商业有限元仿真软件(ANSYS、COMSOL等)均无法较好地建立同时考虑机械接触与背吹气体流动的静电卡盘力学模型,因此本工作未能包括试验数据与仿真结果的对比研究。若能开发出准确建模该系统的仿真方法,则可更好地揭示脱附过程中整个系统内部发生的不易观察、检测的物理过程,获得对晶圆脱附、吸附、乃至整个静电卡盘设计优化的更清楚的认识。