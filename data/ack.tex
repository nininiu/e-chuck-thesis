\begin{ack}
感谢我的指导老师,同时也是我的班主任程嘉老师。从上学期开题以来,程老师一直支持、鼓励我探索具有新颖性、创新性的检测方案,每周耐心地与我讨论项目进展,并引导我与实验室的老师同学和华卓精科公司的工程师交流思想,虚心听取意见建议;整个研究的工作量庞大,在程老师的合理规划下,每一部分的工作才得以有条不紊地按时完成。

在检测平台的设计阶段,我在制造所的吴丹老师的启发下,确定了微力探头的总体设计思路;在制造所的杨开明老师的指导下,在大量产品中选择出了最适合检测平台要求的电动推杆。在此感谢两位老师对我的热情帮助。

加入课题组以来,王珂晟、王兴阔师兄与曹明路师姐一直热情地支持着我的工作,给予了我很多帮助;他们之前做过的工作也为我的检测方案以及试验过程的具体设计提供了很多启发。两位师兄更是抽出了大量时间帮助我完成了最关键的两次试验。

我的研究中使用的静电卡盘从华卓精科公司获得,由北方微电子公司提供。所有机加工件均由华卓精科公司代送外协加工。在此感谢华卓精科许岩经理与张旭光工程师在检测方案设计、检测平台的搭建与调试等阶段给予我的支持与鼓励。还要感谢北方微电子公司的栾大为工程师提供有关待测静电卡盘的准确而详细的技术资料。

在电控系统的具体设计与实现上,清华大学学生创新社(AOI)的张骏琳同学、吴天际学长、以及幻腾智能团队的郎朝见工程师提出了许多宝贵的参考意见,帮助我在较短的时间内解决了电控系统从设计到实现中的各种问题。在此感谢他们无私的帮助。

感谢WIKA中国的汪博工程师在背吹控制系统方面无私提供的宝贵的意见,虽然最后未能使用WIKA产品,这些意见对背吹控制系统最终能够达到检测平台要求起到了至关重要的作用。
\end{ack}
